\section[Method]{Method}
\subsection{Problem Formulation}
\begin{frame}{Problem Formulation}
    \begin{itemize}
        \item<1->Formulate the problem as an energy minimisation problem over a
        densely-connected conditional random field\onslide<2->{, and
        \item iteratively update the training targets to obtain pixelwise object
        segmentations.}
        \onslide<3->{
            \begin{block}{Energy Function}
                \begin{equation}
                    E(\mathbf{f}) = \sum_{i} \psi_u(f_{i})
                                    + \sum_{i < j} \psi_p(f_{i}, f_{j}),
                \end{equation}
                where $\mathbf{f}$ is the pixelwise segmentation, $\psi_u$ is
                the unary potential, and $\psi_p$ is the pairwise potential.
            \end{block}
        }
    \end{itemize}
\end{frame}

\begin{frame}{Problem Formulation}
    \begin{block}{Energy Function}
        \begin{equation*}
            E(\mathbf{f}) = \sum_{i} \psi_u(f_{i})
                            + \sum_{i < j} \psi_p(f_{i}, f_{j}),
        \end{equation*}
        where $\mathbf{f}$ is the pixelwise segmentation, $\psi_u$ is
        the unary potential, and $\psi_p$ is the pairwise potential.
    \end{block}
    \onslide<2->{
        \begin{itemize}
            \item The unary potential is defined as the negative log-likelihood of
                the pixel belonging to the object.
            \item<3-> The pairwise potential penalises label differences for any two
                pixel locations.
        \end{itemize}
    }
\end{frame}

\subsection{CNN}
\begin{frame}{CNN}
    \begin{block}{Energy Function}
        \begin{equation*}
            E(\mathbf{f}) = \sum_{i} \psi_u(f_{i})
                            + \sum_{i < j} \psi_p(f_{i}, f_{j}).
        \end{equation*}
    \end{block}
    \onslide<2->{
        The unary potential term \[
            \psi_u(f_i) = -\log p(y_i | \mathbf{x}; \boldsymbol{\Theta}),
        \] where $p(y_i | \mathbf{x}; \boldsymbol{\Theta})$ is the probability
        of pixel $i$ having label $f_i$.

        We use a CNN to model this probability.
    }
\end{frame}

\begin{frame}{CNN}
    \begin{itemize}
        \item<1->The CNN is trained to predict the probability of a pixel belonging
            to the foreground given the input image.
        \item<2->For each pixel in the image, we pass a $33 \times 33$ patch to
            the CNN, centered at that pixel.
        \item<3->The CNN is trained using the weak annotations provided in the
            form of bounding boxes.
        \item<4->We use the cross-entropy loss function.
    \end{itemize}
\end{frame}

\begin{frame}[fragile]{Code}
    \footnotesize
    \begin{minted}{python}
class CNN(nn.Module):
    def __init__(self):
        super(CNN, self).__init__()
        self.conv1 = nn.Conv2d(1, 32, 5)
        self.bn1 = nn.BatchNorm2d(32)
        self.pool = nn.MaxPool2d(2, 2)
        self.conv2 = nn.Conv2d(32, 32, 3)
        self.bn2 = nn.BatchNorm2d(32)
        self.dropout = nn.Dropout(0.5)
        self.fc1 = nn.Linear(32 * 6 * 6, 256)
        self.fc2 = nn.Linear(256, 2)
        self.init_layers()
        return x
    \end{minted}
\end{frame}

\begin{frame}[fragile]{Code}
    \footnotesize
    \begin{minted}{python}
class CNN(nn.Module):
    def init_layers(self):
        nn.init.kaiming_normal_(self.conv1.weight, nonlinearity='relu')
        nn.init.kaiming_normal_(self.conv2.weight, nonlinearity='relu')
        nn.init.kaiming_normal_(self.fc1.weight, nonlinearity='relu')
        nn.init.kaiming_normal_(self.fc2.weight, nonlinearity='linear')

    def forward(self, x):
        x = self.pool(F.relu(self.bn1(self.conv1(x))))
        x = self.pool(F.relu(self.bn2(self.conv2(x))))
        x = self.dropout(x)
        x = torch.flatten(x, 1)
        x = self.fc1(x)
        x = self.dropout(x)
        x = F.relu(x)
        x = self.fc2(x)
        return x
    \end{minted}
\end{frame}

\begin{frame}{CNN}
    We interrupt training after $N_{\text{epochs per crf}}$ epochs. We then inference
    from the model and use the CRF to refine the segmentation.
\end{frame}

\section{CRF}
\begin{frame}{CRF}
    \begin{block}{Energy Function}
        \begin{equation*}
            E(\mathbf{f}) = \sum_{i} \psi_u(f_{i})
                            + \sum_{i < j} \psi_p(f_{i}, f_{j}),
        \end{equation*}
    \end{block}
    \onslide<2->{
        The pairwise potential is defined as \[
            \psi_p(f_i, f_j) = g(f_i, f_j) [f_i \neq f_j],
        \]
    }
    \onslide<3->{
        where
        \begin{equation}
            g(f_i, f_j) =
                \omega_1\exp\left(-\frac{||p_i - p_j||^2}{2\theta_{\alpha}^2}
                                  -\frac{||I_i - I_j||^2}{2\theta_{\beta}^2}\right)
                + \omega_2\exp\left(-\frac{||p_i - p_j||^2}{2\theta_{\gamma}^2}\right),\label{eq:pairwise}
        \end{equation}
        where $p_i$ and $I_i$ are the spatial and intensity features of pixel $i$.
    }
\end{frame}

\begin{frame}{CRF}
    \begin{equation}
        g(f_i, f_j) =
            \omega_1\exp\left(-\frac{||p_i - p_j||^2}{2\theta_{\alpha}^2}
                              -\frac{||I_i - I_j||^2}{2\theta_{\beta}^2}\right)
            + \omega_2\exp\left(-\frac{||p_i - p_j||^2}{2\theta_{\gamma}^2}\right).
    \end{equation}
    \onslide<2->{
        \begin{itemize}
            \item The first term models the appearance.
            \item<3-> The second term models the smoothness.
        \end{itemize}
    }
\end{frame}

\begin{frame}{CRF}
    \begin{itemize}
        \item<1-> The CRF minimizes the energy function using the mean field approximation.
        \item<2-> Instead of computing $p(\mathbf{x})$ for a labeling
            $\mathbf{x}$ of the image, the mean field approximation computes a
            distribution $q(\mathbf{x})$ that minimizes the KL-divergence
            $\mathbf{D}(q||p)$ among all distributions $q$ that can be expressed
            as a product of independent marginals: \[
                q(\mathbf{x}) = \prod_{i} q_i(x_i).
            \] 
        \item<3-> We use the SimpleCRF library based on the paper by
            Kr\"{a}henb\"{u}hl and Koltun \cite{crf-inference} to do the
            optimisation.
    \end{itemize}
\end{frame}
